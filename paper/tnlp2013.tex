\documentclass[11pt,letterpaper]{article}
\usepackage{acl2013}
\usepackage{times}
\usepackage{latexsym}
\usepackage{amsmath}
\usepackage{mathtools}

\usepackage{graphicx}
\usepackage{algorithm}
\usepackage{algorithmic}

\DeclareMathOperator*{\argmax}{arg\,max}

%\usepackage{authordate1-4}
\usepackage{multirow}

\usepackage{color,soul}
\usepackage[usenames,dvipsnames,svgnames,table]{xcolor}
\newcommand{\Note}[1]{}
\renewcommand{\Note}[1]{\hl{[#1]}}
\newcommand{\NoteSigned}[3]{{\sethlcolor{#2}\Note{#1: #3}}}
\newcommand{\RemoveFF}[1]{\NoteSigned{Remove (FF)}{Crimson}{#1}}
\newcommand{\NoteFF}[1]{\NoteSigned{FF}{LightBlue}{#1}}
\newcommand{\NoteJE}[1]{\NoteSigned{JE}{LightGreen}{#1}}

\setlength\titlebox{6.5cm}    % Expanding the titlebox


\title{A Virtual Manipulative for Learning Log-Linear Models}


\author{
	 Author 1\\
	    XYZ Company\\
	    111 Anywhere Street\\
	    Mytown, NY 10000, USA\\
	    {\tt author1@xyz.org}
	  \And
	  Author 2\\
	    XYZ Company\\
	    111 Anywhere Street\\
	    Mytown, NY 10000, USA\\
	    {\tt author1@xyz.org}
 }
  
\date{}

\begin{document}

\maketitle

\begin{abstract}
Abstract here.
\end{abstract}

\section{Introduction}\label{sec:intro}
except for reading of data files, purely client-side $\Rightarrow$ very easy to set-up;
open-source;
data input format makes it extensible;
individual lessons can be tailored (e.g., hide/show buttons, different tool-tips for lessons)

\section{Model}
Our aim is to teach an intuitive understanding conditional log-linear models. Given $N$ data points $\{( x_i, y_i)\}_{i=1}^N$, we are interested in estimating distributions 
\begin{equation}
\hat{p}(y\ \mid\ x) = \frac{u(x, y)}{\sum_{y'} u(x,y')},
\end{equation}
where $u(x,y)$ represents an unnormalized probability
\begin{eqnarray}
u(x,y) & = & \exp{\left(\vec{\theta}\cdot \vec{f}(x,y)\right)}\\
& = & \exp{\left(\sum_{k=1}^K \theta_k f_k(x,y)\right)}.
\end{eqnarray}

\section{Our Notes}
\NoteFF{These are simply copied from the Google doc titled ``600.465: Maxent Notes.'' This section could be retitled general pedagogical aims, or something of the sort.}
\begin{itemize}
\item If the “striped” feature is predicted to occur less often than it actually does, you should raise its weight.
\item It’s possible to overfit the training data.  Regularization compensates for that and can in fact make you underfit.
\begin{itemize}
\item In particular, weights may zoom off to +infinity or -infinity if a feature is always or never present on the *observed* examples (may need to cook special datasets for this)
\end{itemize}
\item Interactions:
\begin{itemize} 
\item Raising one weight may reduce or reverse the need to raise other weights.  This can be seen by watching the gradient as we slide the slider.
\item Can share features across conditions and this helps regularizer even if likelihood is the same
\item Features that only fire on conditions have no effect on conditional distribution
\item Feature conjunctions: fewer vs. more features
\item Feature that everything/nothing has --- weights go to $\pm \infty$
\item Opposing features, e.g., solid vs striped, where there are only 2 options (or, red vs. blue)
\end{itemize}
\item Likelihood always goes up if you follow gradient
\begin{itemize}
\item gradient = observed - expected count (- regularizer)
\item This is evident in the LL-bar at the top
\end{itemize}
\item LL is maximized when you match the empirical (except for overfitting?)
\item Frequent conditions more influential
\item Some distributions can’t be matched --- but you get generalization
\item The initial setting where all weights = 0 gives the uniform distribution (in each condition).
\begin{itemize}
\item Some further understanding of the entropy view?  (See below.)
\end{itemize}
\end{itemize}

\section{Usability}
\begin{description}
\item[``New Counts'' button] The other use is to help the user experiment with datasets of different sizes, by changing N to scale the counts and then clicking "New counts."
\end{description}

%%%\bibliography{mybib}
\bibliographystyle{acl}

\end{document}
